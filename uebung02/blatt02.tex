%\documentclass[handout]{beamer}
\documentclass{beamer}
\usepackage[ngerman]{babel}
\usepackage[T1]{fontenc}
\usepackage[utf8x]{inputenc}

%\setbeameroption{show notes on second screen=left}
%\setbeameroption{show notes on second screen}
\useinnertheme[shadow=true]{rounded} %grey preview of bullet points
\beamertemplatenavigationsymbolsempty %remove navigation symbols

\usepackage{tikz}
\usetikzlibrary{shapes}
\usetikzlibrary{arrows} 
\tikzstyle{buffer} = [draw,fill=blue!10,minimum width=55,minimum height=20,node distance = 1.5cm]
\tikzstyle{bigsorter} = [draw,fill=blue!30,minimum width=30,minimum height=40,inner sep=0,
regular polygon, regular polygon sides=3,shape border rotate=-90,node distance = 2cm,text width=20,text centered]
\tikzstyle{sorter} = [style=bigsorter,fill=blue!10]
\tikzstyle{mydotted} = [dotted,dash pattern=on 0mm off 2mm,line cap=round, line width=1mm]

\title{Funnelsort}
\subtitle{Algorithm Engineering - Aufgabe 2.3}
\author{Waldemar Smirnow, Michael Stypa}
\institute{Universität Osnabrück}
\date{\today}

\begin{document}
\begin{frame}
  \titlepage
\end{frame}

\begin{frame}
  \frametitle{Tall Cache Assumption}
  \framesubtitle{}
  \begin{center}
    \Large
    $M = \Omega\left(B^2\right)$
  \end{center}
  \begin{itemize}
    \item Cache-Size $M$, Block-Size $B$
    \item $\mathcal{O}\left(g\right)$: wächst nicht wesentlich schneller als $g$
    \item $\Omega\left(g\right)$: wächst nicht wesentlich langsamer als $g$
    \item $\Theta\left(g\right)$: wächst genauso schnell wie $g$
    \item Vorlesungsfolien Externspeicher Folie 8
  \end{itemize}
  $\Rightarrow$ $M$ ist größer als $B^2$ multipliziert mit einem konstanten Faktor
\end{frame}

\begin{frame}
  \frametitle{Funnelsort}
  \framesubtitle{Eigenschaften}
  %prokop
  \begin{itemize}
    \item Vergleichsbasierter Sortier-Algorithmus
    \item Cache Oblivious
    \item Im externen Speichermodel: Sortierung von $N$ Zahlen
      mit cache-size $M$ und block-size $B$
      braucht $\mathcal{O}\left(\frac{N}{B}\log_M{N}\right)$ memory transfers
      unter der tall cache assumption
    \item Es wurde gezeigt dass diese Anzahl an memory transfers
      für vergleichsbasierte Sortierung asymptotisch optimal ist
    \item Ebenfalls asymptotisch optimale Laufzeit in $\mathcal{O}\left(N\log{N}\right)$
  \end{itemize}
\end{frame}

\begin{frame}
  \frametitle{Funnelsort}
  \framesubtitle{Vorgehen}
  \begin{itemize}
    %\item Arbeitet auf zusammenhängenden Arrays
    \item Eingabe-Zahlenliste in $N^{\frac{1}{3}}$ Teillisten
      der Größe $N^{\frac{2}{3}}$ aufteilen und rekursiv soriteren
    \item Alle $N^{\frac{1}{3}}$ sortierten Teillisten zu einer
      sortierten Ergebnisliste mergen mittels $N^{\frac{1}{3}}$-merger
  \end{itemize}
\end{frame}

\begin{frame}
  \frametitle{Funnelsort}
  \framesubtitle{Vergleich mit Mergesort}
  \begin{itemize}
    \item Iteratives Teilen der Eingabe-Zahlenliste
    \item Iteratives Mergen von Teillisten
      in längere sortierte Teillisten
  \end{itemize}
\end{frame}

\begin{frame}
  \frametitle{Funnelsort}
  \framesubtitle{k-Merger}
  \begin{itemize}
  \end{itemize}
\end{frame}

\begin{frame}
  \frametitle{Lazy Funnelsort}
  \framesubtitle{}
  %https://en.wikipedia.org/wiki/Funnelsort#Lazy_Funnelsort
  \begin{itemize}
    \item Buffer müssen nicht immer zur Erschöpfung gefüllt werden
    \item Buffer werden gefüllt wenn sie leer sind
    \item Gleiche Laufzeit und Anzahl an Speicher-Transfers
    \item Anwendungsbereich: Computational Geometry: distribution sweeping
  \end{itemize}
\end{frame}

\begin{frame}
  \frametitle{Funnelsort}
  \framesubtitle{Grafik - Funnel}
  \begin{center}
    \begin{tikzpicture}[every node/.style={scale=0.7},>=stealth']
      \node[bigsorter,scale=6] (sorter) {};

      \node[buffer] (bufferi) [at=(sorter), right=-10] {buffer};
      \node[buffer] (buffer1) [above of=bufferi] {buffer};
      \node[buffer] (buffern) [below of=bufferi] {buffer};

      \node[sorter] (lsorteri) [left of=bufferi, left=-7] {$\mathbb{L}_i$};
      \node[sorter] (lsorter1) [above of=lsorteri, above=0] {$\mathbb{L}_1$};
      \node[sorter] (lsortern) [below of=lsorteri, below=0] {$\mathbb{L}_{\sqrt{k}}$};

      \node[sorter] (rsorter) [right of=bufferi, right=10] {$\mathbb{R}$};

      \draw[->] ([xshift=-1cm, yshift=0.3cm]lsorter1.west) -- ([yshift=0.3cm]lsorter1.west);
      \draw[->] ([xshift=-1cm, yshift=0.0cm]lsorter1.west) -- ([yshift=0.0cm]lsorter1.west);
      \draw[->] ([xshift=-1cm, yshift=-0.3cm]lsorter1.west) -- ([yshift=-0.3cm]lsorter1.west);

      \draw[->] ([xshift=-1cm, yshift=0.3cm]lsorteri.west) -- ([yshift=0.3cm]lsorteri.west);
      \draw[->] ([xshift=-1cm, yshift=0.0cm]lsorteri.west) -- ([yshift=0.0cm]lsorteri.west);
      \draw[->] ([xshift=-1cm, yshift=-0.3cm]lsorteri.west) -- ([yshift=-0.3cm]lsorteri.west);

      \draw[->] ([xshift=-1cm, yshift=0.3cm]lsortern.west) -- ([yshift=0.3cm]lsortern.west);
      \draw[->] ([xshift=-1cm, yshift=0.0cm]lsortern.west) -- ([yshift=0.0cm]lsortern.west);
      \draw[->] ([xshift=-1cm, yshift=-0.3cm]lsortern.west) -- ([yshift=-0.3cm]lsortern.west);

      \draw[->] (lsorter1.east) to [out=east,in=west,looseness=1.5] (buffer1.west);
      \draw[->] (lsorteri.east) to [out=east,in=west] (bufferi.west);
      \draw[->] (lsortern.east) to [out=east,in=west,looseness=1.5] (buffern.west);

      \draw[->] (buffer1.east) to [out=east,in=west,looseness=1.5] ([yshift=0.3cm]rsorter.west);
      \draw[->] (bufferi.east) to [out=east,in=west] ([yshift=0.0cm]rsorter.west);
      \draw[->] (buffern.east) to [out=east,in=west,looseness=1.5] ([yshift=-0.3cm]rsorter.west);

      \draw[->] (rsorter.east) -- ([xshift=1.25cm]rsorter.east);

      \draw[mydotted] ([xshift=-0.5cm, yshift=-0.5cm]lsorter1.west) -- ([xshift=-0.5cm, yshift=0.6cm]lsorteri.west);
      \draw[mydotted] ([xshift=-0.5cm, yshift=-0.5cm]lsorteri.west) -- ([xshift=-0.5cm, yshift=0.6cm]lsortern.west);
    \end{tikzpicture}
  \end{center}
\end{frame}

\begin{frame}
  \frametitle{Funnelsort}
  \framesubtitle{Grafik - 2-merger}
  \begin{center}
    \begin{tikzpicture}[every node/.style={scale=0.7},>=stealth']
      \node[bigsorter,scale=6] (sorter) {};

      \node[buffer] (buffer1) [at=(sorter), right=0, yshift=20] {buffer};
      \node[buffer] (buffer2) [at=(sorter), right=0, yshift=-20] {buffer};

      \node[sorter] (lsorter1) [left of=buffer1, xshift=-30, yshift=50] {$\mathbb{L}_1$};
      \node[sorter] (lsorter2) [left of=buffer2, xshift=-30, yshift=-50] {$\mathbb{L}_2$};

      \node[sorter] (rsorter) [at=(sorter), right=90] {$\mathbb{R}$};

      \draw[->] ([xshift=-1cm, yshift=0.15cm]lsorter1.west) -- ([yshift=0.15cm]lsorter1.west);
      \draw[->] ([xshift=-1cm, yshift=-0.15cm]lsorter1.west) -- ([yshift=-0.15cm]lsorter1.west);
      \draw[->] ([xshift=-1cm, yshift=0.15cm]lsorter2.west) -- ([yshift=0.15cm]lsorter2.west);
      \draw[->] ([xshift=-1cm, yshift=-0.15cm]lsorter2.west) -- ([yshift=-0.15cm]lsorter2.west);

      \draw[->] (lsorter1.east) to [out=east,in=west,looseness=1.5] (buffer1.west);
      \draw[->] (lsorter2.east) to [out=east,in=west,looseness=1.5] (buffer2.west);

      \draw[->] (buffer1.east) to [out=east,in=west,looseness=1.5] ([yshift=0.15cm]rsorter.west);
      \draw[->] (buffer2.east) to [out=east,in=west,looseness=1.5] ([yshift=-0.15cm]rsorter.west);

      \draw[->] (rsorter.east) -- ([xshift=1.25cm]rsorter.east);
    \end{tikzpicture}
  \end{center}
\end{frame}

\begin{frame}
  \frametitle{Funnelsort}
  \framesubtitle{Praxistauglichkeit}
\end{frame}

\end{document}

%Aufgabe 2.3: Funnelsort
%(Lazy) Funnelsort ist ein cache-oblivious Sortieralgorithmus. Wie und warum funktioniert er?
%Was sind seine Laufzeit- und I/O-Garantien? Was kann man über seine Praxistauglichkeit sagen?
