\documentclass{scrartcl}
%\usepackage[ngerman]{babel}
\usepackage[T1]{fontenc}
\usepackage[utf8]{inputenc}

\title{Algorithm Engineering - Übungsblatt 1}
\author{Waldemar Smirnow, Michael Stypa}
\date{\today}

\begin{document}
\maketitle

queue
-first-in-first-out data structure
-first-come-first-server processing of a queue
-performs the function of a buffer
-doesnt have a capacity
-can be empty
-queue underflow
-queue overflow

%klassischen internen Implementierungen des Datentyps Stack besprochen

Queue
\begin{itemize}
  \item enqueue(type $v$) Hängt $v$ an die Queue an
  \item type dequeue() Liefert ältestes Element der Queue und entfernt es
\end{itemize}

Implementierung (optimal im RAM-Modell)
\begin{itemize}
  \item Array + Zeiger auf oberstes Element + Zeiger auf unterstes Element
  \item Zeigerverkettete Liste
\end{itemize}

Anzahl der I/Os (für beliebige Abfolge von Operationen)
\begin{itemize}
  \item O(1) pro Operation
\end{itemize}

%ihr Verhalten im Externspeicher-Setting analysiert

Extern-Queue
\begin{itemize}
  \item Interner Speicher ("Puffer"): Array $J$ der Größe $2B$; restliche Daten extern
  \item $J$ enthält zu jedem Zeitpunkt die $k \leq 2B$ obersten Elemente
\end{itemize}


%eine Modifikation vorgeschlagen, die in diesem Setting beweisbar besser funktioniert
\end{document}
